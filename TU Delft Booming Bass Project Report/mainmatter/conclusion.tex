\chapter{Conclusion}
\label{chapter:conclusion}
\section{Conclusion}
In conclusion, the sound amplifier was successfully designed and built by the team through a structured and methodical approach. Initially, each component was designed separately, supported by calculations and simulations to identify the theoretically optimal values. Once the design phase was complete, the team utilized available lab components to construct each system sub-component. Testing and adjustments were made to improve the performance of the sub-components if needed.

The final result is a well-integrated loudspeaker system capable of producing high-quality audio, as demonstrated by the frequency response in Fig. \ref{fig:frequency_response}. Additionally, the best-performing parts were selected for the final amplifier assembly. The overall audio output was further enhanced when connected to the Booming Bass (\cite{Linkwitz}) extension, achieving greater accuracy by introducing a flatter base.

The power supply group successfully built a fully operational power supply that met all required specifications and requirements. The output voltages were within the specified range, and the ripple voltage remained below 5\%, making it suitable for integration into the amplifier system. The power supply also fully discharges within the specified time of 150 seconds after being disconnected to ensure safe operation.

The power amplifier group also delivered a functional amplifier that met all performance requirements. The corner frequencies observed in the frequency response graph (see \autoref{fig:freq response with phase}) were 20.9Hz and 40.4kHz, with a gain of 27.90dB, demonstrating the amplifier's capability to operate effectively within the desired parameters.

The filter groups completed three distinct filters with cut-off frequencies at 150Hz for the low-pass filter and 1250Hz for the high-pass filter. This design ensures a balanced input to the speaker, contributing to a flat frequency response, as shown in \autoref{fig:frequency_response}. While a maximum difference of 10dB in acoustic power was observed due to a drop in the filter bank's electrical response, this was not considered to be of significant concern.

Python and LT-Spice simulations played a crucial role throughout the project by providing valuable insights into component behavior, allowing the team to make informed decisions and adjustments to optimize the amplifier's performance.

Overall, the project objectives were successfully achieved, resulting in a high-quality end product.
