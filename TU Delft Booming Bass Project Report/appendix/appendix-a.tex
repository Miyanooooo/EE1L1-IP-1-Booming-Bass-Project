\chapter{Source Code}
\label{chapter: Code}


\section{Python Script - Ideal Filter Component Values Calculator}
\begin{lstlisting}[language=Python]
import numpy as np
from itertools import product

# Given cutoff frequencies
f_c1 = 150  # Hz for high-pass
f_c2 = 1250  # Hz for low-pass

# Component value lists
capacitor_values = [
    100e-9, 220e-9, 330e-9, 470e-9, 680e-9, 1e-6, 1.5e-6, 2.2e-6, 2.7e-6, 3.3e-6,
    3.9e-6, 4.7e-6, 5.6e-6, 6.8e-6, 8.2e-6, 10e-6, 15e-6, 22e-6, 33e-6, 47e-6, 100e-6
]

inductor_values = [
    0.15e-3, 0.22e-3, 0.33e-3, 0.47e-3, 0.68e-3, 1e-3, 1.5e-3, 2.2e-3, 3.3e-3, 4.7e-3
]

# Function to calculate cutoff frequency
def calculate_cutoff_frequency(L, C):
    return 1 / (2 * np.pi * np.sqrt(L * C))

# Perform a global search for the best values
best_high_pass = None
best_low_pass = None
best_error = float('inf')

for C_hp, L_hp, C_lp, L_lp in product(capacitor_values, inductor_values, capacitor_values, inductor_values):
    # Calculate cutoff frequencies for high-pass and low-pass
    f_c1_calc = calculate_cutoff_frequency(L_hp, C_hp)
    f_c2_calc = calculate_cutoff_frequency(L_lp, C_lp)

    # Compute the error as the sum of absolute differences
    error = abs(f_c1 - f_c1_calc) + abs(f_c2 - f_c2_calc)

    # Update best values if error is smaller
    if error < best_error:
        best_error = error
        best_high_pass = (C_hp, L_hp, f_c1_calc)
        best_low_pass = (C_lp, L_lp, f_c2_calc)

# Output the best results
print("High-pass filter:")
print(f"C_hp = {best_high_pass[0]:.2e} F, L_hp = {best_high_pass[1]:.2e} H, f_c1_actual = {best_high_pass[2]:.2f} Hz")
print("\nLow-pass filter:")
print(f"C_lp = {best_low_pass[0]:.2e} F, L_lp = {best_low_pass[1]:.2e} H, f_c2_actual = {best_low_pass[2]:.2f} Hz")
\end{lstlisting}
\section{Python Script -  BPF (Band-pass Filter) Frequency Response Graph}
\begin{lstlisting}[language=Python]
    import numpy as np
import scipy.signal as signal
import matplotlib.pyplot as plt

# Function to calculate Q, f0, and Bandwidth (BW) based on given fL and fH
def calculate_parameters(fL, fH):
    f0 = np.sqrt(fL * fH)  # Center frequency (f0)
    BW = fH - fL  # Bandwidth
    Q = f0 / BW  # Quality factor (Q)
    return f0, BW, Q

# Calculate LC components for a second-order low-pass filter
def calculate_LC_components_lowpass(f0, Q):
    L = Q / (2 * np.pi * f0)  # Inductor value in Henry
    C = 1 / (2 * np.pi * f0 * L)  # Capacitor value in Farad
    return L, C

# Calculate LC components for a second-order high-pass filter
def calculate_LC_components_highpass(f0, Q):
    C = Q / (2 * np.pi * f0)  # Capacitor value in Farad
    L = 1 / (2 * np.pi * f0 * C)  # Inductor value in Henry
    return L, C

# Function to design the second-order low-pass and high-pass filters (analog)
def design_filters(fL, fH, Q):
    # Analog low-pass filter transfer function
    omega_H = 2 * np.pi * fH  # Cutoff angular frequency for low-pass filter
    b_lp, a_lp = signal.butter(2, omega_H, btype='low', analog=True)  # Butterworth filter

    # Analog high-pass filter transfer function
    omega_L = 2 * np.pi * fL  # Cutoff angular frequency for high-pass filter
    b_hp, a_hp = signal.butter(2, omega_L, btype='high', analog=True)  # Butterworth filter

    return b_lp, a_lp, b_hp, a_hp

# Function to plot the combined frequency response of the bandpass filter
def plot_combined_response(b_lp, a_lp, b_hp, a_hp, fL, fH):
    # Create a frequency range for plotting (log scale)
    w, h_hp = signal.freqs(b_hp, a_hp, worN=np.logspace(1, 5, 400))  # Frequency response of HPF
    _, h_lp = signal.freqs(b_lp, a_lp, worN=w)  # Frequency response of LPF

    # Multiply the high-pass and low-pass responses to get the bandpass response
    h_bandpass = h_lp * h_hp

    # Plot the bandpass frequency response
    plt.figure(figsize=(10, 6))
    plt.semilogx(w, 20 * np.log10(np.abs(h_bandpass)), label='Bandpass Filter Response')
    plt.title(f"Bandpass Filter Frequency Response (fL={fL} Hz, fH={fH} Hz)")
    plt.xlabel('Frequency [Hz]')
    plt.ylabel('Magnitude [dB]')
    plt.grid(True)
    plt.legend()
    plt.show()

# Main function to design and plot the filter
def main(fL, fH):
    # Calculate f0, BW, and Q
    f0, BW, Q = calculate_parameters(fL, fH)

    # Print the calculated parameters
    print(f"Center frequency (f0): {f0:.2f} Hz")
    print(f"Bandwidth (BW): {BW:.2f} Hz")
    print(f"Quality factor (Q): {Q:.2f}")

    # Calculate the LC components for the low-pass filter
    L_lp, C_lp = calculate_LC_components_lowpass(f0, Q)
    print(f"Low-pass filter components:")
    print(f"Inductor (L) for low-pass filter: {L_lp:.6f} H")
    print(f"Capacitor (C) for low-pass filter: {C_lp:.6e} F")

    # Calculate the LC components for the high-pass filter
    L_hp, C_hp = calculate_LC_components_highpass(f0, Q)
    print(f"\nHigh-pass filter components:")
    print(f"Inductor (L) for high-pass filter: {L_hp:.6f} H")
    print(f"Capacitor (C) for high-pass filter: {C_hp:.6e} F")

    # Design the low-pass and high-pass filters
    b_lp, a_lp, b_hp, a_hp = design_filters(fL, fH, Q)

    # Plot the combined frequency response of the bandpass filter
    plot_combined_response(b_lp, a_lp, b_hp, a_hp, fL, fH)

# Example usage:
if __name__ == "__main__":
    fL = 150  # Low-pass cutoff frequency in Hz
    fH = 1250  # High-pass cutoff frequency in Hz
    main(fL, fH)

\end{lstlisting}


\section{Python - Phase Calculator}

\label{calculation of phase}
\begin{lstlisting}[language=Python]

def phase_calculation(frequency):
    f = frequency
    om = f * 2 * math.pi

    C_1 = 166.5 * (10 ** -9)
    R_2 = 47000
    C_a = 4.05 * (10 ** -9)
    R_a = 987
    R_b = 81800
    C_x = 968 * (10 ** -9)
    R_c = 1.98 * (10 ** 6)

    a = C_1 * R_2
    b = R_a * C_a
    d = a * b
    f = R_b * C_x
    k = (om ** 2) * a * R_c * C_x
    t = a+b+f
    q = (a*f) + k
    r = (f*(a+b)) + d
    
    Re_1 = ((om**4)*q*r) - ((om**2)*q)
    Re_2 = ((om**2)*a*t)-((om**4)*a*d*f)
    Re = Re_1 + Re_2

    Im_1 = (om*a) - ((om**3)*a*r)
    Im_2 = ((om**3)*q*t) - ((om**5)*d*f*q)
    Im = Im_1 + Im_2
    phase = np.degrees(np.atan(Im/Re))
    return phase

\end{lstlisting}
\newpage
\section{Python - Ripple and Average Voltage Simulations}
\label{sec:ripvoltcodesim}
\begin{lstlisting}[language=Python]
import pandas as pd

def ripple(avg, max):
    return (max-avg)/avg*100


datas = [(".\\19 ohm.txt",19),
    	(".\\38 ohm.txt",38),
    	(".\\76 ohm.txt",76),
    	(".\\no load.txt",None)]

for data in datas:
    file = data[0]
    resistance = data[1]
    
    data = pd.read_csv(file, delimiter="\t")
    #Only take the last part of the simulation to avoid the waveform when the capacitors are still being loaded
    t = data["time"][len(data["time"])//3*2:]
    v = data["V(n001)"][len(data["V(n001)"])//3*2:]
    i = data["I(R2)"][len(data["I(R2)"])//3*2:]

    maxV = v.max()
    minV = v.min()
    avgV = (maxV + minV)/2
    meanI = i.mean()
    
    if resistance == None:
        print(f"Using {file} (Power supply is not loaded)")
    else:
        print(f"Using {file} (Power supply is loaded with a {resistance} Ohm resistor)")

    print(f"    Load resistance: {resistance}")   
    print(f"    Average Voltage: {avgV:.3}V")
    print(f"    Average Current: {meanI:.3}A")
    print(f"    Ripple: {ripple(avgV, maxV):.3}%")
\end{lstlisting}

\section{Python - Ripple and Average Voltage Measurements}
\label{sec:ripvoltcodemea}

\begin{lstlisting}[language=Python]
import pandas as pd

def ripple(avg, max):
    return (max-avg)/avg*100
# Tuples are formatted as (filename,resistance)
datas = [(".\\19 ohm\\F0009CH1.CSV",19),
    	(".\\38 ohm\\F0007CH1.CSV",38),
    	(".\\76 ohm\\F0011CH1.CSV",76),
    	(".\\No load\\F0008CH1.CSV",None)]
for data in datas:
    file = data[0]
    resistance = data[1]
    
    measured_data = pd.read_csv(file, usecols=[3,4])
    t = measured_data["d"]
    v = measured_data["e"]

    maxV = v.max()
    minV = v.min()
    avgV = (maxV + minV)/2
    if resistance == None:
        print(f"Using {file} (Power supply is not loaded)")
    else:
        print(f"Using {file} (Power supply is loaded with a {resistance} Ohm resistor)")
    print(f"    Load resistance: {resistance}")   
    print(f"    Average Voltage: {avgV:.3}V")
    if resistance == None:
        print(f"    Average Current: 0.00A")
    else:
        print(f"    Average Current: {avgV/resistance:.3}A")
    print(f"    Ripple: {ripple(avgV, maxV):.3}%")

\end{lstlisting}


\DeclareTCBListing{mintedbox}{O{}m!O{}}{%
  breakable=true,
  listing engine=minted,
  listing only,
  minted language=#2,
  minted style=default,
  minted options={%
    linenos,
    gobble=0,
    breaklines=true,
    breakafter=,,
    fontsize=\small,
    numbersep=8pt,
    #1},
  boxsep=0pt,
  left skip=0pt,
  right skip=0pt,
  left=25pt,
  right=0pt,
  top=3pt,
  bottom=3pt,
  arc=5pt,
  leftrule=0pt,
  rightrule=0pt,
  bottomrule=2pt,
  toprule=2pt,
  colback=bg,
  colframe=orange!70,
  enhanced,
  overlay={%
    \begin{tcbclipinterior}[h!]
    \fill[orange!20!white] (frame.south west) rectangle ([xshift=20pt]frame.north west);
    \end{tcbclipinterior}},
  #3}
